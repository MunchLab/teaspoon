While I\textquotesingle{}m sure this workflow will change as we go, here are some links to get you started.


\begin{DoxyItemize}
\item A nice introduction to git\+: \href{http://gitimmersion.com/}{\tt Git Immersion}
\item Other packages to install in order to get full functionality out of teaspoon\+:
\begin{DoxyItemize}
\item \href{https://github.com/Ripser/ripser}{\tt Ripser}. Code by Ulrich Bauer for computing persistent homology of a point cloud or distance matrix.
\item \href{http://people.maths.ox.ac.uk/nanda/perseus/index.html}{\tt Perseus}. Code by Vidit Nanda for computing persistent homology of point clouds, cubical complexes, and distance matrices.
\item \href{https://bitbucket.org/grey_narn/hera}{\tt Hera}. Code by Michael Kerber, Dmitriy Morozov, and Arnur Nigmetov for computing bottleneck and Wasserstein distances.
\end{DoxyItemize}
\item Please put issues in the \href{https://gitlab.msu.edu/TSAwithTDA/teaspoon/issues}{\tt issue tracker}
\end{DoxyItemize}

\section*{Basic workflow}

When you\textquotesingle{}re going to start messing with something, create a branch for it. This will likely just be done on your local machine. When you think it\textquotesingle{}s ready to be merged into the master branch, go to the merge request page on gilab and submit your merge request.

Disscussions, comments, and updates can be done on the gitlab merge request page.

\section*{Questions, comments, and other issues}

...should be sent to \href{mailto:muncheli@msu.edu}{\tt Liz Munch}. 